In this section, we describe our experimental setup that includes
applications, testbed information. We also discussed about the Charm++
language with which these applications are written and also elaborate about
its dynamic load balancing feature.

\subsection{Applications}
We used two applications, namely, Jacobi2D \& LeanMD\cite{leanmd} from the
Charm++ test repository.  We have manually modified these applications so that
to get the precise timing measurements.  These applications have different CPU
and memory usage.
\begin{description}
\item[Jacobi2D] \hspace{1mm}

A 5-point stencil \textbf{memory-bound} application that computes the transmission of heat over a
discretized 2D grid. The global 2D grid is divided into smaller blocks that are
processed in parallel. It is an iterative application where all processors
synchronize at the end of each iteration. As is the case in a stencil
computation, each grid point is the average of the neighboring 5 points.
For example, the new value for element X is the current
value of X plus the current values of its left, right, top, and
bottom neighbors. 
\begin{center}       
          T    \\
        L X R  \\
          B    \\
\end{center}       
X'  = (X + L + R + T + B) / 5.0

Neighboring blocks communicate the ghost layers with each other so that
averaging computations are done for all cells inside each block. This
application is implemented in Charm++ using a 2D chare array.

\item [LeanMD] \hspace{1mm}

LeanMD \cite{leanmd} is a \textbf{computationally intensive} molecular dynamics application.
This benchmark simulates atomic interaction based on Lennard-Jones potential.


LeanMD is a molecular dynamics simulation application written in Charm++ for
PetaFLOPs class supercomputers. It is being developed as the next generation of
NAMD \cite{namd}, one of the parallel
applications winning the Gordon Bell Award in SC2002. NAMD, as a
state-of-the-art parallel molecular dynamics application that is also written
in Charm++, has already been proven to be able to scale to 3000 processors.
However, it is not ready for next generation parallel machines with hundreds of
thousands, or even millions, of processors due to the limited parallelism
exploited in the application.  Clearly, it requires a new parallelization
strategy that can break up the problem in a more fine-grained manner to
effectively distribute work across the extremely large number of processors.
With that outlook in mind, LeanMD is being developed as an experimental code.

LeanMD computes the interaction forces based on Lennard-Jones forces amongst
particles in a 3D space. It does not include any long range force calculation.
The object decomposition is achieved using a scheme similar to NAMD. The 3D
space is divided into hyper-rectangles, called cells or patches in NAMD’s
nomenclature, each containing a subset of particles. A compute object is
responsible for the force calculations between each pair of cells. In each
computation of the application, each cell sends its particle data to all
computes objects attached to it and receives the updates from those computes
objects. This mini-application is implemented using Charm++ where the set of
cells and compute objects are represented by chare arrays.
\end{description}

\subsection {Testbed}
Our testbed is a 60-node Dell PowerEdge R620 cluster installed at the
Department of Computer Science, University of Illinois at Urbana-Champaign.
Each node is an Intel Xeon E5-2620 Sandy-bridge server with 6 physical cores @
2GHz, 2-way SMT with 16GB of DRAM.  The Intel Sandy Bridge processor family
supports on board power measurement and capping through the Running Average
Power Limit (RAPL) interface \cite{rapl}.  The Sandy Bridge architecture has
four power planes: Package (PKG), Power Plane 0 (PP0), Power Plane 1 (PP1) and
DRAM. RAPL is implemented using a series of Machine Specifics Registers (MSRs)
  which can be accessed to get power readings for each power plane. RAPL
  supports power capping PKG, PP0 and DRAM power planes by writing into the
relevant MSRs. The package power\footnote{Package corresponds to the processor
  chip that hosts processing cores, caches and memory controller} for our
testbed can be capped in the range 25W to 95W (71 integer power levels) while
the memory power can be capped between 8W to 35W (28 integer power levels). The
average base power per node for our cluster was 38 watts.  The base power was
measured using the in-built power meters on the Power Distribution Unit (PDU)
  that powers our cluster.  

In our experiments we will \textbf{NOT} be capping the
memory power as our work is focused on studying the heterogeneity that comes up at
lower CPU power. The effect of lower memory power on heterogeneity is not
discussed.

\subsection {Charm++ and Load Balancing} 
For this research, we used Charm++ programming paradigm which supports
dynamic object migration to improve performance of a parallel
application\cite{KaleCharm}.  It relies on techniques such as processor
virtualization and over-decomposition (having more work units than the
    number of cores) to improve performance via adaptive overlap of
computation and communication and data-driven execution.  Charm++ gives the
freedom to the programmer to define program into multiple grain size objects
which can be migrated across the cores. The programmer need not make the
application core aware. This multiple objects defined by the programmer is
moved around during program execution by adaptive runtime system not only
for load balancing purposes but also for communication optimization and
fault tolerance. Load balancer keeps the statistics of all the migratable
objects for effective load balancing act\cite{appBalancer99}.  The runtime
system provides load balancing strategies that can account for different
application characteristics.  Application programmers can provide their own
implementation of load balancers based on the characteristics of the
application and the ecosystem under which it is run.
     
     In this study we first observed the effect of an existing Charm++ load
     balancer, RefineLB, on mitigating the node heterogeneity at lower power
     caps.  RefineLB uses an algorithm that strives to reduce the cost of
     migration by moving only a few objects from overloaded processors to
     underloaded ones so that load of all processors gets close to the
     average.  This is done by examining every object on an overloaded
     processor and looking for the best choice of an underloaded processor
     to migrate the object to\cite{Zheng}.   
     
     In this work, we are planning to develop a power aware  load balancer
     in the sense that it will use aggressive approach to mitigate the load
     imbalance at lower power cap. By aggressive we mean it will take lesser
     time than the existing load balancer to achieve good amount of load
     balance.




