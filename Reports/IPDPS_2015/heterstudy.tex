\subsection{Heterogeneity metric}
Our first contribution is to show that at lower power capping values, different
nodes show prominent differences in their runtime performance.  In this work,
      we use Intel's power\_gov library\cite{power_gov} that in turn uses RAPL
      \cite{rapl} to cap power of memory and package\footnote{Package
        corresponds to the processor chip that hosts processing cores, caches
          and memory controller} subsystems.


We define heterogeneity at a given power cap in terms of idle waiting times of
the cores at that power cap.  We defined the idle waiting time of a core at a
given power cap in the following two ways: 
\\ 
\\
\noindent At a given power cap, let  $t_{i}, 1\leq i \leq C$  be the overall
execution time of the core $i$ for a particular application and $T$ be the
total execution time of the application.  

\begin{description}
  \item[Average idle waiting time, $I_{av}$] \hfill \\ 
    \begin{equation} \label{eq:1}
      I_{av} = \frac{\displaystyle\sum\limits_{i=0}^P (\displaystyle\max_{1\leq j \leq C} ( t_{j} ) - t_{i})}{C}
    \end{equation}
  \item[Max idle waiting time, $I_{m}$] \hfill \\
    \begin{equation} \label{eq:2}
      I_{m} = \displaystyle\max_{1\leq j \leq C} ( t_{j} ) -  \displaystyle\min_{1\leq i \leq C} ( t_{i} )
    \end{equation}
\end{description}

Figure \ref{fig:1} shows that at lower power caps the
idle waiting times (Equations \eqref{eq:1} and \eqref{eq:2}) are having higher values as
compared to those at higher power caps. 

\begin{figure}
\centering
\begin{tabular}{c c}
  \scalebox{0.45}{
    \begin{tikzpicture}
    \begin{axis}[
     xlabel=  Power Cap Values (W),
     ylabel = Average Idle Time (secs),
     ymax=12, ymin=0, xmax=60, xmin=23,
     x tick label style={black},
     grid=both
     ]
    \addplot table [x=POWER, y= WOLB_AV_IDLE]{data.dat};
    \addlegendentry {Without LB}
    \end{axis}
    \end{tikzpicture}
  }
& 
  \scalebox{0.45}{
    \begin{tikzpicture}
    \begin{axis}[
    xlabel=  Power Cap Values (W),
    ylabel = Max Idle Time (secs),
    ymax=31, ymin=6, xmax=60, xmin=23,
    x tick label style={black},
    grid=both
    ]
    \addplot table [x=POWER, y= WOLB_MAX_IDLE_P]{data.dat};
    \addlegendentry {Without LB}
    \end{axis}
    \end{tikzpicture}
  }
\\
\end{tabular}
\caption{Behavior of idle waiting (both average and max) times at lower power caps}
\label{fig:1}
\end{figure}

\subsection{Testbed}
Our Testbed is a 60-node Dell PowerEdge R620 cluster installed at the
Department of Computer Science, University of Illinois at Urbana-Champaign.
Each node is an Intel Xeon E5-2620 Sandy-bridge server with 6 physical cores @
2GHz, 2-way SMT with 16GB of DRAM.  The Intel Sandy Bridge processor family
supports on board power measurement and capping through the Running Average
Power Limit (RAPL) interface \cite{rapl}.  The Sandy Bridge architecture has
four power planes: Package (PKG), Power Plane 0 (PP0), Power Plane 1 (PP1) and
DRAM. RAPL is implemented using a series of Machine Specifics Registers (MSRs)
  which can be accessed to get power readings for each power plane. RAPL
  supports power capping PKG, PP0 and DRAM power planes by writing into the
relevant MSRs. The package power\footnote{Package corresponds to the processor
  chip that hosts processing cores, caches and memory controller} for our
Testbed can be capped in the range 23W to 95W (73 integer power levels) while
the memory power can be capped between 8W to 35W (28 integer power levels). The
average base power per node for our cluster was 38 watts.  The base power was
measured using the in-built power meters on the Power Distribution Unit (PDU)
  that powers our cluster.  

In our experiments we will \textbf{NOT} be capping the
memory power as our work is focused on studying the heterogeneity that comes up at
lower CPU power. The effect of lower memory power on heterogeneity is not
discussed.



\subsection{Applications used}
We used two applications, namely, Jacobi2D \& LeanMD\cite{leanmd} from the
Charm++ test repository.  We have manually instrumented these applications so that
we get the precise timing measurements.  These applications have different CPU
and memory usage.

\begin{itemize}
\item \textbf{Jacobi2D: } A 5-point stencil \textbf{memory-bound} application
that computes the transmission of heat over a discretized 2D grid. The global
2D grid is divided into smaller blocks that are processed in parallel. It is an
iterative application where all processors synchronize at the end of each
iteration. As is the case in a stencil computation, each grid point is the
average of the neighboring 5 points.  For example, the new value for element X (i.e. X')
is the current value of X plus the current values of its left, right, top, and
bottom neighbors.  

\begin{center}       
T    \\ L X R  \\ B    \\ 
X'  = (X + L + R + T + B) / 5.0
\end{center}

Neighboring blocks communicate the ghost layers with each other so that
averaging computations are done for all cells inside each block. This
application is implemented in Charm++ using a 2D chare array.

\item \textbf{LeanMD: } LeanMD \cite{leanmd} is a \textbf{computationally
  intensive} molecular dynamics application.  This benchmark simulates atomic
  interaction based on Lennard-Jones potential.

LeanMD is a molecular dynamics simulation application written in Charm++ for
PetaFLOPs class supercomputers. It is being developed as the next generation of
NAMD \cite{namd}, one of the parallel
applications winning the Gordon Bell Award in SC2002. NAMD, as a
state-of-the-art parallel molecular dynamics application that is also written
in Charm++, has already been proven to be able to scale to 3000 processors.
However, it is not ready for next generation parallel machines with hundreds of
thousands, or even millions, of processors due to the limited parallelism
exploited in the application.  Clearly, it requires a new parallelization
strategy that can break up the problem in a more fine-grained manner to
effectively distribute work across the extremely large number of processors.
With that outlook in mind, LeanMD is being developed as an experimental code.

LeanMD computes the interaction forces based on Lennard-Jones forces amongst
particles in a 3D space. It does not include any long range force calculation.
The object decomposition is achieved using a scheme similar to NAMD. The 3D
space is divided into hyper-rectangles, called cells or patches in NAMD’s
nomenclature, each containing a subset of particles. A compute object is
responsible for the force calculations between each pair of cells. In each
computation of the application, each cell sends its particle data to all
computes objects attached to it and receives the updates from those computes
objects. This mini-application is implemented using Charm++ where the set of
cells and compute objects are represented by chare arrays.
\end{itemize}


\subsection{Charm++ \& Load Balancing}
For this research, we used Charm++ programming paradigm which supports dynamic
object migration to improve performance of a parallel
application\cite{KaleCharm}.  It relies on techniques such as processor
virtualization and over-decomposition (having more work units than the number
    of cores) to improve performance via adaptive overlap of computation and
communication and data-driven execution.  Charm++ gives the freedom to the
programmer to define program into multiple grain size objects which can be
migrated across the cores. The programmer need not make the application core
aware. This multiple objects defined by the programmer is moved around during
program execution by adaptive runtime system not only for load balancing
purposes but also for communication optimization and fault tolerance. Load
balancer keeps the statistics of all the migratable objects for effective load
balancing act\cite{appBalancer99}.  The runtime system provides load balancing
strategies that can account for different application characteristics.
Application programmers can provide their own implementation of load balancers
based on the characteristics of the application and the ecosystem under which
it is run.


\subsection{Observations}

\begin{figure*}
\centering
\begin{tabular}{cc}
  \scalebox{0.75}{
    \begin{tikzpicture}
    \begin{axis}[
     xlabel=  Power Cap Values (W),
     ylabel = Total Execution Time (secs),
     ymax=33, ymin=7, xmax=60, xmin=23,
     x tick label style={black},
     grid=both
     ]
    \addplot table [x=POWER, y= WOLB_MAX_TIME]{data.dat};
    \addlegendentry {Without LB}
    \end{axis}
    \end{tikzpicture}
  }
& 
  \scalebox{0.75}{
    \begin{tikzpicture}
    \begin{axis}[
     xlabel=  Power Cap Values (W),
     ylabel = Max Idle Time (secs),
     ymax=31, ymin=6, xmax=60, xmin=23,
     x tick label style={black},
     grid=both
     ]
    \addplot table [x=POWER, y= WOLB_MAX_IDLE_P]{data.dat};
    \addlegendentry {Without LB}
    \end{axis}
    \end{tikzpicture}
  }
\\
\qquad (a) & \quad (b) \\
  \scalebox{0.75}{
    \begin{tikzpicture}
    \begin{axis}[
     xlabel=  Power Cap Values (W),
     ylabel = Average Idle Time (secs),
     ymax=12, ymin=0, xmax=60, xmin=23,
     x tick label style={black},
     grid=both
     ]
    \addplot table [x=POWER, y= WOLB_AV_IDLE]{data.dat};
    \addlegendentry {Without LB}
    \end{axis}
    \end{tikzpicture}
  }
& 
  \scalebox{0.75}{
    \begin{tikzpicture}
    \begin{axis}[
     xlabel=  Core Ids,
     ylabel= Total Execution Time (secs),
     ymax=34, ymin=2, xmax=60, xmin=0,
     x tick label style={black},
     grid=both
     ]
    \addplot table [x=PE, y= WOLB]{data1.dat};
    \addlegendentry {Without LB}
    \end{axis}
    \end{tikzpicture}
  } \\
\qquad (c) & \quad (d) \\
\end{tabular}
\caption{Experiment to highlight the heterogeneity involved among the nodes
  when Jacobi2D application is run on 10 cluster nodes with different power caps
    ranging from 23W to 60W.  (a) Total Execution Time vs Power (b) Max Idle
    Time vs Power (c) Average Idle Time vs Power (d) Heterogeneity at Power cap
    of 23W} \label{fig:init_heter} \end{figure*}

The following observations are made on the execution times of Jacobi2D application
run on 10 cluster nodes ( of the aforementioned cluster) under different power caps ranging from 
23W to 60W. For this particular experiment we have not turned on any load balancer.

Figure\ref{fig:init_heter}(a) shows how the execution times varied under
different power caps.  As we can see, execution time of the application
increases as we move from higher power to lower power caps. We see an
exponential increase of execution time as we move to the lower power values.
This is partly due to reduction in frequency as the power power level goes down
and hence increased execution times. But we also have heterogeneity factor
contributing to the increased execution times which will explore next. 

Figure\ref{fig:init_heter}(b) shows how “Max Idle Time” metric varies under
 different power values.  As we can see under lower power regions the max idle
 times increases exponentially. This means the difference in execution times
 vary a lot under lower power regions. This suggests that there is a big margin
 between the processors which finish its load quickly and the processor which
 finished the last. In the next part let’s see how average idle times varies
 under lower power capping regions. This metric gives a better understanding of
 how the idle time varies across the nodes and not just the difference between
 the best and the worst performer unlike this graph.


Figure \ref{fig:init_heter}(c) how “Average Idle Time” metric varies under
different power values. As we can see the average idle time varies exponentially
under lower power capping regions. This shows that most or all processors
exhibit different execution times and there by having different idle times and
hence we see increased average idle times under lower power regions. This
further establishes the fact that heterogeneity is present across the nodes and
not just only with few nodes. Next we will show the extent of
heterogeneity among the processors. 


Figure \ref{fig:init_heter}(d) show how the execution time has varied
across the processors at 23W. As we see we have huge difference in
execution times among the processors. The difference in these execution times
as we have seen is best captured by “max idle time” and “average idle time”
graphs. The execution time of the processors within a node is more or less a
constant. The execution times vary only among the PEs across the nodes. This
shows that heterogeneity holds good across the nodes and not within the same
node.
